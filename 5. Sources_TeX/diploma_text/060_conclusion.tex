% !TeX spellcheck = ru_RU
% !TEX root = vkr.tex

\section*{Заключение}

В ходе работы была достигнута поставленная цель: разработан инструмент для автоматизации профориентации на основе предсказания кода Голланда по неполным результатам психометрических тестов личности. Для выполнения цели были решены следующие задачи:
% [noitemsep, topsep=0pt, parsep=0pt, partopsep=0pt]
\begin{enumerate}
    \item Разработаны и реализованы математические модели модуля восстановления пропусков результатов психометрических тестов: \emph{MICE}, маски, ансамбли по набору заполненных тестов и метод мягкой импутации, который в сочетании с PSO-ансамблем регрессоров показал наибольший C-индекс:~$10.74$.

    \item Реализованы подходы к определению кода Голланда: многоцелевая регрессия (\emph{multioutput}, \emph{chain}), классификация (\emph{multiclass}, \emph{multilabel}, \emph{label powerset}), списочное ранжирование.

    \item Разработан модуль формирования взвешенного ансамбля моделей на основе методов подбора весов: метод роя частиц (PSO), равные веса, частичный перебор по сетке, вектор Шэпли, квадратичная оптимизация, генетический алгоритм и координатный спуск.
    
    \item Проведен сравнительный анализ моделей предсказания кодов Голланда; лучшие результаты у моделей линейного блендинга с оптимизацией весов моделей методом роя частиц:
      \begin{itemize}[noitemsep, topsep=0pt, parsep=0pt, partopsep=0pt]
        \item ансамбль \emph{multioutput}‑регрессоров: Lasso- и пошаговая регрессии, CatBoost, ExtraTrees ($\text{C‑index}~=~11.663$);
        \item ансамбль \emph{multilabel}‑классификаторов: kNN, SVM, логистическая Lasso-регрессия, XGBoost, LightGBM и др. ($\text{C‑index}~=~11.625$);
        \item лучшая базовая модель~--- L1-регрессия со множественными выходами ($\text{C‑index}~=~11.175$);
        \item показано превосходство классических методов машинного обучения над нейросетевыми в данной задаче.
      \end{itemize}

    \item Создан прототип инструмента для определения профориентационных предпочтений на основе R Shiny.
\end{enumerate}

Исходный код всего проекта представлен в GitHub-репозитории\footnote{\quad GitHub: Предсказание кода Голланда (RIASEC) по результатам психометрических тестов личности. URL: \url{https://github.com/ExP98/Diploma_Holland} (дата обращения: 17.05.2025).}. Отдельные аспекты вычислительного эксперимента, связанные с решением задач регрессии и классификации, ранее были представлены на XXVIII Международной конференции по мягким вычислениям и измерениям SCM'2025.

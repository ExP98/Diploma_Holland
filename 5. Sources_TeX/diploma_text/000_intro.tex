% !TeX spellcheck = ru_RU
% !TEX root = vkr.tex


\section*{Введение}

Многие аспекты успешности человека обусловлены корректным определением карьерного пути, который соответствует его личностным предпочтениям~\cite{Presti}. От того, насколько успешно человек определил свою социально-профессиональную направленность, зависит удовлетворенность человека своей работой~\cite{Aydıntan, Cannas, Medgyesi}. Карьерному самоопределению уделяются значительные ресурсы, в том числе со стороны государства~\cite{FZ489}, чьи усилия направляются в том числе на профессиональное образование людей, которые по окончании обучения не работают по своей специальности вследствие отсутствия интереса к выбранной профессии. С определением своего будущего с профессиональной точки зрения сталкивается любой выпускник школы и вуза, безработный, работник, не удовлетворенный текущей работой или уже находящийся в процессе её смены, — для всех этих категорий людей встает вопрос об определении своих профориентационных предпочтений. Золотым стандартом является глубинное интервью с экспертом, который поможет выявить сильные и слабые стороны личности. Однако этот подход является ресурсозатратным, и для автоматизации процесса профориентации используются дистанционные способы карьерного консультирования~\cite{Pordelan, Westman}, в том числе профориентационные тесты, доступные для прохождения онлайн и не требующие большого количества времени для прохождения~\cite{vk_psychotests}.

Одним из инструментов для определения профессиональных интересов уже более полувека является модель Дж. Голланда RIASEC~\cite{Holland1959}, которая предполагает, что профессиональные предпочтения являются отражением характера человека, его базовых черт. Так, вводятся шесть типов социально-профессиональной направленности личности (кодов Голланда): реалистический (Realistic, R), исследовательский (Investigative, I), артистический (Artistic, A), социальный (Social, S), предприимчивый (Enterprising, E) или традиционный (Conventional, C). Коды могут определяться при помощи тестирования, в котором в результате попарного сравнения профессий численно оцениваются указанные шесть типов направленности личности~\cite{Rezapkina}. Для сравнения профессиональных профилей личности (кодов Голланда) между собой используется C-индекс как мера сходства (конгруэнтности). Существует множество вариаций данного теста~\cite{Chu}, результаты которых коррелированы, однако не определяют друг друга однозначно. Кроме того, такие тесты часто не учитывают быстро изменяющуюся конъюнктуру рынка профессий, а также культурные и социо-экономические различия респондентов~\cite{Chu, Hoff, Nye}. Возникает актуальная задача определения кода Голланда по альтернативным данным.

В настоящее время существуют исследования, показывающие взаимосвязь кода Голланда индивида с его социально-демографическими признаками~\cite{Bogacheva}, цифровым следом в социальных сетях (его сообщения, посты, фото; в исследованиях могли быть использованы иные психометрические тесты)~\cite{Chekalev, Ivaschenko, Stoliarova, Oliseenko, Stankevich, Titov, Basaran}. Однако основное внимание исследователей направлено на изучение взаимосвязи факторов модели Голланда с факторами других психометрических тестов~\cite{Mason, Hurtado, Batista, Usslepp, Schuerger, Yamashita, Martins}. Для анализа взаимосвязей различных опросных инструментов исследователи используют методы регрессионного анализа и статистические тесты~\cite{Hurtado, Ivaschenko, Stoliarova}, моделирование структурными уравнениями (\emph{SEM})~\cite{Martins, Chu} и модели машинного обучения~\cite{Usslepp, Silva, Song, Bogacheva, Oliseenko, Stankevich, Titov, Basaran}.

Несмотря на наличие работ, в которых предсказывается результат одного психометрического теста на основе другого теста или на основе некоторых признаков личности (например, комментариев, постов и фото пользователей социальной сети), до сих пор нет инструментов, позволяющих по результатам одного или комбинации сразу нескольких популярных психометрических тестов (\enquote{Большая пятёрка}, Кеттелла, Айзенка, Леонгарда, Шварца) предсказывать код Голланда. Востребован инструмент, в котором результаты тестов могли бы быть предоставлены частично, который бы в меньшей степени зависел от изменяющейся конъюнктуры рынка профессий, от культурных и экономических различий респондентов, то есть позволял определять профессиональный профиль личности в условиях подобной неполноты информации. В результате, пользователь мог бы получить информацию о своих профориентационных предпочтениях без прохождения теста Голланда. Кроме того, даже при прохождении последнего подобный инструмент может использоваться для уточнения результатов теста Голланда и установления его непротиворечивости в соответствии с результатами других исследований.

Данная работа призвана закрыть этот пробел: с помощью современных методов машинного обучения на основе собранных данных по результатам прохождения одного или нескольких указанных психометрических тестов личности устанавливаются взаимосвязи между результатами тестов и предсказывается код Голланда, соответствующий профессиональным предпочтениям личности. Предсказание кода Голланда может быть решено такими методами машинного обучения, как многоцелевая регрессия, классификация или ранжирование~\cite{Bishop}; для улучшения качества прогноза базовые модели могут объединяться в ансамбли~\cite{ZhangC, Bischl}. Возможность предсказания по одному или нескольким тестам влечет необходимость восстановления результатов тестов, которые не были пройдены.

Новизна результатов исследования состоит в создании нового программного комплекса, обеспечивающего автоматизацию процесса профориентации на основе предсказания кода Голланда. Теоретическая значимость заключается в использовании уникальной комбинации различных психометрических тестов при разработке новых моделей машинного обучения для определения взаимосвязи тестов и кода Голланда. Практическая значимость~--- разработка прототипа программного модуля автоматизации оценки профессиональной направленности по психологическому профилю личности.

\renewcommand{\gr}[1]{\gradientcelld{#1}{2.038}{2.06}{2.14}{high}{mid}{low}{70}}

\begin{table}
    \captionsetup{skip=-0.5ex, belowskip=2pt}
    \footnotesize
    \setlength{\tabcolsep}{0pt}
    \centering
    \caption{Весовые коэффициенты моделей и RMSE при разных методах подбора весов для ансамблей на восстановленных данных}
    \label{tab:impute_ens_rmse}
    \begin{tabular*}{0.95\textwidth}{@{\extracolsep{\fill}}
        >{\raggedright\arraybackslash}m{4.35cm}|
        *{5}{>{\centering\arraybackslash}m{1.51cm}}
        |>{\centering\arraybackslash}m{1.9cm}
      @{}}
      \toprule
      \multicolumn{1}{c|}{\textbf{Подбор весов}}
        & \multicolumn{5}{c|}{\textbf{Веса моделей}}
        & \textbf{RMSE} \\
      \cmidrule(lr){2-6}
      \multicolumn{1}{c|}{}
        & \textbf{Lasso L1}
        & \textbf{Пошагов.}
        & \textbf{LightGBM}
        & \textbf{Случ. лес}
        & \textbf{kNN}
        & \\ 
      \midrule
      Метод роя частиц (PSO)     & 0.001 & 0.481 & 0.038 & 0.475 & 0.005 & \gr{2.038} \\
      Частичный перебор по сетке & 0.000 & 0.500 & 0.000 & 0.500 & 0.000 & \gr{2.043} \\
      Генетический алгоритм (GA) & 0.281 & 0.369 & 0.109 & 0.189 & 0.052 & \gr{2.044} \\
      Координатный спуск         & 0.019 & 0.422 & 0.067 & 0.305 & 0.187 & \gr{2.052} \\
      Вектор Шэпли (Shap)        & 0.247 & 0.185 & 0.206 & 0.179 & 0.183 & \gr{2.063} \\
      Равные веса всех моделей   & 0.200 & 0.200 & 0.200 & 0.200 & 0.200 & \gr{2.064} \\
      Квадратичная оптимиз. (QP) & 0.050 & 0.000 & 0.390 & 0.007 & 0.553 & \gr{2.114} \\
      \bottomrule
    \end{tabular*}
    \vspace{0.5em}
    \begin{minipage}{\textwidth}
      \scriptsize
      \textit{\hspace*{1.5em}Обозначения:\\
      \hspace*{2.5em}Пошагов.~--- пошаговая регрессия,\\
      \hspace*{2.5em}Случ. лес~--- случайный лес (Random Forest),\\
      \hspace*{2.5em}kNN~--- метод k-ближайших соседей}
    \end{minipage}
\end{table}

\renewcommand{\gr}[1]{\gradientcelld{#1}{2.025}{2.05}{2.2}{high}{mid}{low}{70}}

\begin{table}
    \captionsetup{skip=-0.5ex, belowskip=2pt}
    \centering
    \small
    \setlength{\tabcolsep}{0pt}
    \caption{Сравнение методов подбора весов ансамбля регрессионных моделей по RMSE}
    \label{tab:regr_ensembles_rmse}
    \begin{tabular*}{0.95\textwidth}{@{\extracolsep{\fill}}
        >{\raggedright\arraybackslash}m{5.5cm}|
        *{4}{>{\centering\arraybackslash}m{2.2cm}}
      @{}}
      \toprule
        \multicolumn{1}{>{\centering\arraybackslash}m{5.5cm}|}{\textbf{Метод подбора весов}}
          & \multicolumn{2}{c}{\textbf{Multioutput}}
          & \multicolumn{2}{c}{\textbf{Chained}} \\
        \cmidrule(lr){2-3}\cmidrule(lr){4-5}
          & \textbf{все модели} & \textbf{топ-5}
          & \textbf{все модели} & \textbf{топ-5} \\
      \midrule
      Равные веса всех моделей        & \gr{2.052} & \gr{2.038} & \gr{2.052} & \gr{2.032} \\
      Вектор Шэпли (Shap)             & \gr{2.047} & \gr{2.035} & \gr{2.046} & \gr{2.033} \\
      Частичный перебор по сетке      & \gr{2.052} & \gr{2.026} & \gr{2.045} & \gr{2.035} \\
      Квадратичная оптимизация (QP)   & \gr{2.109} & \gr{2.093} & \gr{2.111} & \gr{2.070} \\
      Генетический алгоритм (GA)      & \gr{2.035} & \gr{2.036} & \gr{2.044} & \gr{2.032} \\
      Метод роя частиц (PSO)          & \gr{2.049} & \gr{2.031} & \gr{2.065} & \gr{2.026} \\
      Координатный спуск              & \gr{2.097} & \gr{2.048} & \gr{2.089} & \gr{2.040} \\
      \midrule
      Лин. регрессии с регуляризацией
        & \multicolumn{2}{c}{Линейная регрессия}
        & \multicolumn{2}{c}{\gr{2.111}} \\
      L1, L2, LightGBM, CatBoost, RF
        & \multicolumn{2}{c}{Линейная регрессия}
        & \multicolumn{2}{c}{\gr{2.091}} \\
      \bottomrule
    \end{tabular*}
    \begin{minipage}{\textwidth}
      \footnotesize
      \textit{\hspace*{1.5em}Обозначения:\\
      \hspace*{2.5em}топ-5~--- подбор весов только для топ-5 моделей согласно метрике,\\
      \hspace*{2.5em}L1 и L2~--- Lasso- и Ridge-модели регрессии, RF~--- случайный лес}
    \end{minipage}
\end{table}

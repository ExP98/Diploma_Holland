\renewcommand{\g}[1]{\gradientcelld{#1}{7}{10.5}{11.2}{low}{mid}{high}{70}}

\begin{table}
    \captionsetup{skip=-0.5ex, belowskip=1pt}
    \footnotesize
    \centering
    \caption{Сравнение базовых регрессионных моделей по C-индексу}
    \label{tab:regr_res}
    \setlength{\tabcolsep}{0pt}
    \begin{tabular*}{0.9\textwidth}{@{\extracolsep{\fill}}
          >{\raggedright\arraybackslash}m{5.2cm}|
          *{4}{>{\centering\arraybackslash}m{2.08cm}}
        @{}}
      \toprule
      \multicolumn{1}{>{\centering\arraybackslash}m{5.2cm}|}{\textbf{Модель}}
        & \multicolumn{2}{c}{\textbf{Multioutput}}
        & \multicolumn{2}{c}{\textbf{Chained}} \\
      \cmidrule(lr){2-3}\cmidrule(lr){4-5}
        & \textbf{без PCA} & \textbf{PCA}
        & \textbf{без PCA} & \textbf{PCA} \\
      \midrule
      Регрессия Lasso (L1)      & \g{11.175} & \g{10.887} & \g{11.175} & \g{11.150} \\
      ExtraTrees                & \g{10.700} & \g{11.100} & \g{10.625} & \g{10.825} \\
      Регрессия Ridge (L2)      & \g{10.988} & \g{10.537} & \g{11.062} & \g{10.412} \\
      Метод опорных векторов    & \g{10.713} & \g{10.950} & \g{10.713} & \g{10.950} \\
      Пошаговая регрессия       & \g{10.605} & \g{10.905} & \g{10.600} & \g{10.905} \\
      CatBoost                  & \g{10.688} & \g{10.812} & \g{10.688} & \g{10.812} \\
      Случайный лес             & \g{10.625} & \g{10.475} & \g{10.812} & \g{10.588} \\
      LightGBM                  & \g{10.750} & \g{10.425} & \g{10.750} & \g{10.425} \\
      k-ближайших соседей (kNN) & \g{10.525} & \g{10.400} & \g{10.525} & \g{10.400} \\
      XGBoost                   & \g{9.164}  & \g{9.729}  & \g{9.162}  & \g{9.725}  \\
      Базовая константная       & \g{9.000}  & \g{9.000}  & \g{9.000}  & \g{9.000}  \\
      \midrule
      TabPFN                    & \g{10.562} &            &            &            \\
      MLP (BN, DropOut, регуляризация) & \g{10.462} &     &            &            \\
      Многослойный перцептрон (MLP)    & \g{10.275} &     &            &            \\
      \bottomrule
    \end{tabular*}
    % \vspace{0.75em}
    % \begin{minipage}{\textwidth}
    %   \scriptsize
    %   \textit{\hspace*{3em} Обозначения: PCA~--- метод главных компонент (уменьшение размерности), BN~--- пакетная нормализация}
    % \end{minipage}
\end{table}


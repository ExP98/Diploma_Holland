\renewcommand{\g}[1]{\gradientcelld{#1}{9}{10.25}{11}{low}{mid}{high}{70}}

\begin{table}
    \footnotesize
    \setlength{\tabcolsep}{0pt}
    \centering
    \caption{Весовые коэффициенты моделей и C-индекс при разных методах подбора весов для ансамблей на восстановленных данных}
    \label{tab:impute_ens}
    \begin{tabular*}{0.95\textwidth}{@{\extracolsep{\fill}} 
        >{\raggedright\arraybackslash}m{4.35cm}|
        *{5}{>{\centering\arraybackslash}m{1.5cm}}|
        >{\centering\arraybackslash}m{2cm}
      @{}}
        \toprule
        \multicolumn{1}{c|}{\textbf{Подбор весов}} 
          & \multicolumn{5}{c|}{\textbf{Веса моделей}} 
          & \textbf{С-индекс}\\
        \cmidrule(lr){2-6}
        \multicolumn{1}{c|}{}  
          & \textbf{Lasso L1} 
          & \textbf{Пошагов.} 
          & \textbf{LightGBM} 
          & \textbf{Случ. лес} 
          & \textbf{kNN} 
          & \textbf{} \\
        \midrule
        Метод роя частиц (PSO)     & 0.001 & 0.481 & 0.038 & 0.475 & 0.005 & \g{10.740} \\
        Частичный перебор по сетке & 0.000 & 0.500 & 0.000 & 0.500 & 0.000 & \g{10.657} \\
        Генетический алгоритм (GA) & 0.281 & 0.369 & 0.109 & 0.189 & 0.052 & \g{10.401} \\
        Координатный спуск         & 0.019 & 0.422 & 0.067 & 0.305 & 0.187 & \g{10.245} \\
        Квадратичная оптимиз. (QP) & 0.050 & 0.000 & 0.390 & 0.007 & 0.553 & \g{10.065} \\
        Равные веса всех моделей   & 0.200 & 0.200 & 0.200 & 0.200 & 0.200 & \g{10.053} \\
        Вектор Шэпли (Shap)        & 0.247 & 0.185 & 0.206 & 0.179 & 0.183 & \g{10.047} \\
        \bottomrule
    \end{tabular*}
        \vspace{0.5em}
        \begin{minipage}{\textwidth}
        \scriptsize
        \textit{\hspace*{1.5em}Обозначения:\\
        \hspace*{2.5em}Пошагов. ~--- пошаговая регрессия,\\
        \hspace*{2.5em}Случ. лес~--- случайный лес (Random Forest),\\
        \hspace*{2.5em}kNN~--- метод k-ближайших соседей}
    \end{minipage}
\end{table}

% !TeX spellcheck = ru_RU
% !TEX root = vkr.tex

% Опциональные добавления используемых пакетов. Вполне может быть, что они вам не понадобятся, но в шаблоне приведены примеры их использования.
\usepackage{tikz} % Мощный пакет для создание рисунков, однако может очень сильно замедлять компиляцию
\usetikzlibrary{decorations.pathreplacing,calc,shapes,positioning,tikzmark}

% Библиотека для TikZ, которая генерирует отдельные файлы для каждого рисунка
% Позволяет ускорить компиляцию, однако имеет свои ограничения
% Например, ломает пример выделения кода в листинге из шаблона
% \usetikzlibrary{external}
% \tikzexternalize[prefix=figures/]

\newcounter{tmkcount}

\tikzset{
    use tikzmark/.style={
            remember picture,
            overlay,
            execute at end picture={
                    \stepcounter{tmkcount}
                },
        },
    tikzmark suffix={-\thetmkcount}
}


%% Таблицы
\usepackage{tabularx}
\usepackage{makecell}
\usepackage{booktabs}
\usepackage{multirow}
\usepackage{multicol}
\usepackage{ltablex}
\usepackage{longtable}
\usepackage{array}


% Для названий стоит использовать \textsc{}
\newcommand{\OCaml}{\textsc{OCaml}}
\definecolor{eclipseGreen}{RGB}{63,127,95}

\makeatletter

%%% Обязательные пакеты
%% Beamer
\usepackage{beamerthemesplit}
\usetheme{SPbGU}
\beamertemplatenavigationsymbolsempty
\usepackage{appendixnumberbeamer}

%% Локализация
\usepackage{fontspec}
\setmainfont{CMU Serif}
\setsansfont{CMU Sans Serif}
\setmonofont{CMU Typewriter Text}
% \setmonofont{Fira Code}[Contextuals=Alternate,Scale=0.9]
% \setmonofont{Inconsolata}

\newfontfamily\cyrscfont{Inconsolata}[
  SmallCapsFeatures={Letters=SmallCaps},
  Scale=MatchLowercase
]
\DeclareRobustCommand{\textsc}[1]{{\cyrscfont\addfontfeatures{Letters=SmallCaps}#1}}

\usepackage{polyglossia}
\setmainlanguage{russian}
\setotherlanguage{english}

%% Графика
\usepackage{pdfpages} % Позволяет вставлять многостраничные pdf документы в текст
\usepackage{fancyvrb}

\makeatother

\usepackage[autostyle]{csquotes} % Правильные кавычки в зависимости от языка
\usepackage{totcount}
\usepackage{setspace}
\usepackage{amsmath, amsfonts, amssymb, amsthm, mathtools} % "Адекватная" работа с математикой в LaTeX

\usepackage[labelsep=period,            % вместо ':' ставить '.'
            justification=centering,
            singlelinecheck=false
           ]{caption} % Настройка подписей "не текста"
\usepackage{subcaption} % Подписи для разделенного "не текста"
\addto\captionsrussian{\renewcommand{\figurename}{Рисунок}}

\usepackage{pgf}
\usepackage{colortbl}
\usepackage[table]{xcolor}
\usepackage{highlight}
\definecolor{low}{HTML}{ef3b2c}
\definecolor{mid}{HTML}{fff7f7}
\definecolor{high}{HTML}{66FF66}
\newcommand{\g}[1]{\gradientcelld{#1}{6}{10.5}{11.5}{low}{mid}{high}{70}}
\newcommand{\gr}[1]{\gradientcelld{#1}{2.0}{2.1}{2.8}{high}{mid}{low}{70}}

\newcommand{\h}[1]{%
  \pgfmathparse{#1>0.05}%
  \ifnum\pgfmathresult=1
    \cellcolor{high!50}#1%
  \else
    #1%
  \fi
}

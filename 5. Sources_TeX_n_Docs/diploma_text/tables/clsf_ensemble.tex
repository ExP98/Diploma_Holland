\renewcommand{\g}[1]{\gradientcelld{#1}{8}{10.8}{11.7}{low}{mid}{high}{70}}

\begingroup
    \fontsize{8pt}{9pt}\selectfont
    \setlength{\tabcolsep}{0pt}
    \begin{table}
      \centering
      \caption{Сравнение методов подбора весов ансамбля классификаторов}
      \label{tab:clsf_ensemble}
      \begin{tabular*}{\textwidth}{@{\extracolsep{\fill}}
        >{\raggedright\arraybackslash}m{8cm} 
        *{3}{>{\centering\arraybackslash}m{2.75cm}} @{}}
        \toprule
        \textbf{Метод подбора весов}
          & \textbf{Multiclass}
          & \textbf{Multilabel}
          & \textbf{Label Powerset} \\
        \midrule
        Равные веса всех моделей      & \g{10.663} & \g{10.888} & \g{10.563} \\
        Вектор Шэпли (Shap)           & \g{10.563} & \g{11.038} & \g{10.525} \\
        Частичный перебор по сетке    & \g{11.213} & \g{11.488} & \g{11.525} \\
        Квадратичная оптимизация (QP) & \g{10.488} & \g{10.638} & \g{10.650} \\
        Генетический алгоритм (GA)    & \g{11.263} & \g{11.313} & \g{11.213} \\
        Метод роя частиц (PSO)        & \g{11.263} & \g{11.625} & \g{11.525} \\
        Координатный спуск            & \g{11.200} & \g{11.275} & \g{10.425} \\
        \bottomrule
      \end{tabular*}
    \end{table}
\endgroup

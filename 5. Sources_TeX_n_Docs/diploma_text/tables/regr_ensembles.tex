\renewcommand{\g}[1]{\gradientcelld{#1}{9}{11.1}{11.8}{low}{mid}{high}{70}}

\begingroup
  \scriptsize
  \begin{table}
    \centering
    \caption{Сравнение методов подбора весов ансамбля регрессионных моделей}
    \label{tab:regr_ensembles}
    \begin{tabular*}{\textwidth}{@{\extracolsep{\fill}}
        >{\raggedright\arraybackslash}m{8cm}|
        *{4}{>{\centering\arraybackslash}m{1.77cm}}
      @{}}
      \toprule
        \multicolumn{1}{>{\centering\arraybackslash}m{8cm}|}{\textbf{Метод подбора весов}} 
          & \textbf{Multi\-output} 
          & \textbf{Mult. топ-5} 
          & \textbf{Chain} 
          & \textbf{Chain топ-5} \\
      \midrule
      Равные веса всех моделей       & \g{11.063} & \g{11.088} & \g{11.050} & \g{11.013} \\
      Вектор Шэпли (Shap)            & \g{11.050} & \g{11.138} & \g{11.138} & \g{11.050} \\
      Частичный перебор по сетке     & \g{11.550} & \g{11.388} & \g{11.538} & \g{11.325} \\
      Квадратичная оптимизация (QP)  & \g{10.588} & \g{10.463} & \g{10.738} & \g{10.813} \\
      Генетический алгоритм (GA)     & \g{11.500} & \g{11.550} & \g{11.300} & \g{11.563} \\
      Метод роя частиц (PSO)         & \g{11.600} & \g{11.663} & \g{11.613} & \g{11.613} \\
      Координатный спуск             & \g{11.188} & \g{11.225} & \g{11.288} & \g{11.413} \\
      \midrule
      Линейные регрессии с регуляризацией 
        & \multicolumn{2}{c}{Линейная регр.} 
        & \multicolumn{2}{c}{\g{10.887}} \\
      Lasso, Ridge, LightGBM, CatBoost, RF 
        & \multicolumn{2}{c}{Линейная регр.} 
        & \multicolumn{2}{c}{\g{10.688}} \\
      \bottomrule
    \end{tabular*}
    \vspace{0.5em}
    \begin{minipage}{\textwidth}
      \small
      \textit{Обозначения:\\
      \hspace*{1em}Mult.~--- Multioutput, топ-5~--- подбор весов только для топ-5 моделей по C-индексу}
    \end{minipage}
  \end{table}
\endgroup

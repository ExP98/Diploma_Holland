\renewcommand{\g}[1]{\gradientcelld{#1}{9}{10.25}{11}{low}{mid}{high}{70}}

\begin{table}
  \setlength{\tabcolsep}{0pt}
  \centering
  \caption{Весовые коэффициенты моделей и C-индекс при разных методах подбора весов для ансамблей на восстановленных данных}
  \label{tab:impute_ens}
  \begin{tabular*}{\textwidth}{@{\extracolsep{\fill}} 
        >{\raggedright\arraybackslash}m{1.5cm}
        *{5}{c}
        >{\centering\arraybackslash}m{1.9cm}
      @{}}
    \toprule
    \textbf{Подбор} 
      & \multicolumn{5}{c}{\textbf{Веса моделей}} 
      & \textbf{C-}\\
    \cmidrule(lr){2-6}
    \textbf{весов} 
      & \textbf{Lasso L1} 
      & \textbf{Пошаг.} 
      & \textbf{LightGBM} 
      & \textbf{Случ. лес} 
      & \textbf{kNN} 
      & \textbf{индекс} \\
    \midrule
    PSO    & 0.001 & 0.481 & 0.038 & 0.475 & 0.005 & \g{10.740} \\
    Grid   & 0.000 & 0.500 & 0.000 & 0.500 & 0.000 & \g{10.657} \\
    GA     & 0.281 & 0.369 & 0.109 & 0.189 & 0.052 & \g{10.401} \\
    Спуск  & 0.019 & 0.422 & 0.067 & 0.305 & 0.187 & \g{10.245} \\
    QP     & 0.050 & 0.000 & 0.390 & 0.007 & 0.553 & \g{10.065} \\
    Равные & 0.200 & 0.200 & 0.200 & 0.200 & 0.200 & \g{10.053} \\
    Шэпли  & 0.247 & 0.185 & 0.206 & 0.179 & 0.183 & \g{10.047} \\
    \bottomrule
  \end{tabular*}
  \vspace{0.5em}
  \begin{minipage}{\textwidth}
    \small
    \textit{Обозначения:\\
    \hspace*{1em}PSO~--- метод роя частиц, Grid~--- частичный перебор по сетке,\\
    \hspace*{1em}GA~--- генетический алгоритм, спуск~--- координатный спуск,\\
    \hspace*{1em}QP~--- квадратичная оптимизация, пошаг. ~--- пошаговая регрессия,\\
    \hspace*{1em}случ. лес~--- случайный лес (Random Forest), kNN~--- метод k-ближайших соседей}
  \end{minipage}
\end{table}


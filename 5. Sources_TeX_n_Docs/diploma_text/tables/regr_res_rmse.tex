\newcommand{\grmse}[1]{\gradientcelld{#1}{2.0}{2.1}{2.8}{high}{mid}{low}{70}}

\begingroup
    \fontsize{8pt}{9pt}\selectfont
    \begin{table}
      \centering
      \caption{Сравнение регрессионных моделей, метрика RMSE}
      \label{tab:regr_res_rmse}
        \setlength{\tabcolsep}{2pt}
        \begin{tabular*}{\textwidth}{@{\extracolsep{\fill}} 
          >{\raggedright\arraybackslash}m{6.25cm}  
          | *{4}{>{\centering\arraybackslash}m{2.35cm}}
        @{}}
      \toprule
          \textbf{Модель} 
            & \textbf{Multi\-output}      
            & \textbf{Мult. PCA}    
            & \textbf{Chain}   
            & \textbf{Chain PCA} \\
        \midrule
        Регрессия Lasso (L1)      & \grmse{2.018} & \grmse{2.036} & \grmse{2.018} & \grmse{2.030} \\
        Линейная регрессия (OLS)  & \grmse{2.155} & \grmse{2.019} & \grmse{2.155} & \grmse{2.019} \\
        Регрессия Ridge (L2)      & \grmse{2.025} & \grmse{2.037} & \grmse{2.028} & \grmse{2.044} \\
        Пошаговая регрессия       & \grmse{2.094} & \grmse{2.027} & \grmse{2.094} & \grmse{2.027} \\
        CatBoost                  & \grmse{2.044} & \grmse{2.096} & \grmse{2.044} & \grmse{2.096} \\
        Случайный лес             & \grmse{2.069} & \grmse{2.131} & \grmse{2.070} & \grmse{2.133} \\
        LightGBM                  & \grmse{2.074} & \grmse{2.128} & \grmse{2.074} & \grmse{2.128} \\
        Метод опорных векторов    & \grmse{2.100} & \grmse{2.101} & \grmse{2.100} & \grmse{2.101} \\
        ExtraTrees                & \grmse{2.100} & \grmse{2.150} & \grmse{2.112} & \grmse{2.152} \\
        kNN                       & \grmse{2.162} & \grmse{2.151} & \grmse{2.162} & \grmse{2.151} \\
        Constant baseline         & \grmse{2.308} & \grmse{2.308} & \grmse{2.308} & \grmse{2.308} \\
        XGBoost                   & \grmse{2.317} & \grmse{2.314} & \grmse{2.317} & \grmse{2.314} \\
        \midrule
        TabPFN                    & \grmse{2.056} &            &            &            \\
        MLP (BatchNorm, DropOut, регул-я) & \grmse{2.143} &            &            &            \\
        MLP                       & \grmse{2.442} &            &            &            \\
        \bottomrule
      \end{tabular*}
      \vspace{0.75em}
      \begin{minipage}{\textwidth}
          \small
          \textit{Обозначения:\\
          \hspace*{1em}Mult.~--- multioutput, предсказание переменных независимо друг от друга,\\
          \hspace*{1em}Chain~--- предсказание выходных переменных по цепочке,\\
          \hspace*{1em}PCA~--- метод главных компонент (уменьшение размерности)}
      \end{minipage}
    \end{table}
\endgroup
